% -*- TeX -*-
\documentclass{article}

% --- preamble from template ---
\usepackage[T2A]{fontenc}
\usepackage[utf8]{inputenc}
\usepackage[russian]{babel}
\usepackage{amsmath}
\usepackage{amssymb}
\usepackage{amsthm}
\usepackage{mathrsfs}
\usepackage{hyperref}

\usepackage{dan2e}

\makeatletter
\@ifundefined{@Russian}{\newif\if@Russian \@Russiantrue}{}
\@ifundefined{@TVP}     {\newif\if@TVP    \@TVPfalse}{}
\makeatother

\theoremstyle{definition}
\newtheorem{defi}{Определение}
\theoremstyle{plain}
\newtheorem{remark}{Пример}
\newtheorem{theorem}{Теорема}
\newtheorem{OldTheorem}{Теорема}
\renewcommand{\theOldTheorem}{\Alph{OldTheorem}}

\newtheorem{Theorem}{Теорема}
\renewcommand{\theTheorem}{\arabic{theorem}$^\prime$}

% --- журнал‑специфичные метаданные (ДОЛЖНЫ стоять **до** \begin{document}) ---
\begin{document}
\Volume{505}
\Year{2025}
\Pages{46--49}

\udk{517.54}

\title{Оценка общих и специальных знаний в больших языковых моделях для русского языка посредством воспроизведения энциклопедических статей}

\author{Д.\,А.~Григорьев\Addressmark[1]\Emailmark[1], Д.\,И.~Чернышев\Addressmark[1]\Emailmark[2]}

\Addresstext[1]{Московский государственный университет им.~М.\,В.~Ломоносова, Москва, Россия}

\Emailtext[1]{dagrig14@yandex.ru}
\Emailtext[2]{chdanorbis@yandex.ru}

\markboth{Д.\,А.~Григорьев, Д.\,И.~Чернышев}{Оценка общих и специальных знаний в больших языковых моделях}

\presentedby{Представлено кем-то}

% Даты публикации (пример)
\dateA{16.08.2025}
\dateB{20.08.2025}
\dateC{31.08.2025}

% -------------------------------------------------------


\maketitle

\begin{abstract}
Работа исследует методы сжатия художественных текстов с помощью языковых моделей и предлагает улучшенные подходы для точного реферирования в условиях ограниченного контекста.
\end{abstract}

\begin{keywords}
LLM, реферирование, литература, книги, краткий пересказ
\end{keywords}

% ================= ВВЕДЕНИЕ =================
\section*{Введение}
\subsection*{Реферирование художественной литературы}
Автоматическое реферирование текста — одна из ключевых задач в области обработки естественного языка. Суть этой задачи заключается в создании информативной аннотации исходного текста с сохранением основного смысла содержания. В последние годы, с появлением больших языковых моделей, резко возрос интерес к автоматизации реферирования в самых разных жанрах текстов, включая художественные произведения. В отличие от научных, новостных или технических текстов, художественные произведения характеризуются высокой степенью стилистической и семантической сложности. Нелинейность повествования, образность, метафоричность и стилистические приёмы делают задачу написания краткого содержания особенно трудоёмкой. Ограниченное контекстное окно современных моделей дополнительно осложняет работу с длинными произведениями.

Теоретически автоматическое реферирование может выполняться двумя основными способами: извлекающим (выбор ключевых фрагментов текста) и абстрактивным (генерация нового текста на основе содержания оригинала). Для художественной литературы более уместен второй подход, поскольку он позволяет передать смысл и стиль произведения, не нарушая его целостности.

% ================= НАБОР ДАННЫХ =================
\section*{Набор данных}
На момент начала исследования не существовало открытых и репрезентативных корпусов, предназначенных специально для задачи реферирования художественных текстов на русском языке. Для экспериментов был сформирован собственный набор данных, включающий:
\begin{itemize}
  \item более 5~000 пользовательских пересказов с ресурса «Народный Брифли»;
  \item исходные произведения из электронной библиотеки LibRuSec (публичное достояние или тексты с разрешения правообладателей);
  \item очищенные версии аннотаций, полученные с помощью LLM~Meta--Llama~3--70B--Instruct.
\end{itemize}
Тексты аннотаций проходили автоматическую очистку от HTML‑тегов, комментариев и служебных пометок. Затем к каждой аннотации добавлялся оригинальный текст произведения. Такая пара «текст – аннотация» использовалась для тренировки и тестирования методов реферирования.

% ================= ОЦЕНИВАНИЕ =================
\section*{Оценивание методов}
На момент начала исследования в русскоязычном сегменте отсутствовали открытые и репрезентативные наборы данных, специально предназначенные для задачи автоматического реферирования художественных текстов.
В отличие от аналогичных англоязычных проектов, которые уже включали крупные корпуса литературных произведений с высококачественными аннотациями, для русскоязычных текстов подобные ресурсы не были представлены. 
Это существенно затрудняло объективное тестирование и сравнение эффективности методов автоматического реферирования в условиях художественных произведений на русском языке.

С целью проведения экспериментов и оценки различных подходов к генерации аннотаций был создан собственный корпус, состоящий из художественных текстов и соответствующих кратких пересказов. 
В качестве источника для аннотаций был выбран ресурс «Народный Брифли»~\cite{Briefly} — платформа, где пользователи публикуют краткие пересказы литературных произведений. 
Несмотря на вариативность качества и стиля пользовательских аннотаций и наличие нерелевантной информации, такой как учебные вопросы или редакторские замечания, после тщательной предварительной обработки удалось получить достаточно надёжный и чистый набор данных.

Художественные тексты были отобраны из электронной библиотеки LibRuSec — одного из крупнейших русскоязычных ресурсов художественной литературы, содержащего свыше 400 тысяч текстов. 
Отбор произведений осуществлялся на основании наличия аннотаций на выбранном ресурсе. Каждый текст проходил автоматическую предварительную обработку: удалялась метаинформация (например, заголовки, описания глав и технические вставки), 
после чего текст форматировался в единый стандартизированный вид, подходящий для дальнейшего использования в моделях.
Важно отметить, что при создании корпуса использовались только тексты, находящиеся в общественном достоянии или распространяемые свободно с разрешения правообладателей, что обеспечивает соблюдение требований авторского права.

Для иллюстрации структуры сформированного корпуса ниже представлен пример пары «фрагмент художественного текста — соответствующая аннотация».
% ================= ПРИМЕНЕНИЕ МЕТОДОВ =================
\section*{Применение методов}
Экспериментальная часть включает проверку базовых стратегий (иерархическая, итеративная, «чертёжная») и двух усовершенствований.

\subsection*{Влияние предварительной очистки}
Очистка исходного текста от технических артефактов положительно сказалась на ROUGE--L и BERTScore для моделей Meta\-Llama~3\-70B и Ru\-adapt\-Qwen2.5\-32B\-Pro\-Beta, повысив также полноту покрытия ключевых вопросов.

\subsection*{Сравнение базовых методов}
Иерархический и итеративный подходы продемонстрировали сопоставимое качество (ROUGE--L\,$\approx$\,0.48, BERTScore\,$\approx$\,0.70), заметно превосходя псевдо‑генерацию без доступа к исходному тексту.

\subsection*{Иерархический метод с фильтрацией узлов}
Добавление фильтра по семантической близости (порог~0.85) позволило убрать избыточные фрагменты и ускорить генерацию на длинных текстах в среднем на~15\,\%.

\subsection*{«Чертёжный» метод с кластеризацией вопросов}
Кластеризация эмбеддингов вопросов алгоритмом \textit{k}-means и последующее сэмплирование 10--30 вопросов из каждого кластера уменьшили число обращений к модели и ускорили работу метода в~1.22 раза без потери качества.

% -------------------------------------------------------

%-------РЕФЕРЕНСЫ---------
\begin{thebibliography}{99}
\bibitem{Briefly}
\textit{Народный Брифли.}  
Электронная библиотека кратких пересказов литературных произведений.  
\url{https://wiki.briefly.ru/} (дата обращения: 16.07.2025).

\end{thebibliography}

\renewcommand\refname{References}


\begin{thebibliography}{99}
\bibitem{Briefly_e}
\textit{Народный Брифли.}  
Электронная библиотека кратких пересказов литературных произведений.  
\url{https://wiki.briefly.ru/} (дата обращения: 16.07.2025).

\end{thebibliography}

\end{document}