% -*- TeX -*-
\documentclass{article}

% --- preamble from template ---
\usepackage[T2A]{fontenc}
\usepackage[utf8]{inputenc}
\usepackage[russian]{babel}
\usepackage{amsmath}
\usepackage{amssymb}
\usepackage{amsthm}
\usepackage{mathrsfs}
\usepackage{hyperref}
\usepackage{booktabs}
\usepackage{multirow}

\usepackage{dan2e}

\makeatletter
\@ifundefined{@Russian}{\newif\if@Russian \@Russiantrue}{}
\@ifundefined{@TVP}     {\newif\if@TVP    \@TVPfalse}{}
\makeatother

\theoremstyle{definition}
\newtheorem{defi}{Определение}
\theoremstyle{plain}
\newtheorem{remark}{Пример}
\newtheorem{theorem}{Теорема}
\newtheorem{OldTheorem}{Теорема}
\renewcommand{\theOldTheorem}{\Alph{OldTheorem}}

\newtheorem{Theorem}{Теорема}
\renewcommand{\theTheorem}{\arabic{theorem}$^\prime$}

% --- журнал‑специфичные метаданные (ДОЛЖНЫ стоять **до** \begin{document}) ---
\begin{document}
\Volume{505}
\Year{2025}
\Pages{46--49}

\udk{517.54}

\title{Оценка общих и специальных знаний в больших языковых моделях для русского языка посредством воспроизведения энциклопедических статей}

\author{Д.\,А.~Григорьев\Addressmark[1]\Emailmark[1], Д.\,И.~Чернышев\Addressmark[1]\Emailmark[2]}

\Addresstext[1]{Московский государственный университет им.~М.\,В.~Ломоносова, Москва, Россия}

\Emailtext[1]{dagrig14@yandex.ru}
\Emailtext[2]{chdanorbis@yandex.ru}

\markboth{Д.\,А.~Григорьев, Д.\,И.~Чернышев}{Оценка общих и специальных знаний в больших языковых моделях}

\presentedby{Представлено кем-то}

% Даты публикации (пример)
\dateA{16.08.2025}
\dateB{20.08.2025}
\dateC{31.08.2025}

% -------------------------------------------------------


\maketitle

\begin{abstract}
Работа исследует методы сжатия художественных текстов с помощью языковых моделей и предлагает улучшенные подходы для точного реферирования в условиях ограниченного контекста.
\end{abstract}

\begin{keywords}
LLM, реферирование, литература, книги, краткий пересказ
\end{keywords}

% ================= ВВЕДЕНИЕ =================
\section*{Введение}
\subsection*{Реферирование художественной литературы}
Автоматическое реферирование текста — одна из ключевых задач в области обработки естественного языка. Суть этой задачи заключается в создании информативной аннотации исходного текста с сохранением основного смысла содержания. В последние годы, с появлением больших языковых моделей, резко возрос интерес к автоматизации реферирования в самых разных жанрах текстов, включая художественные произведения. В отличие от научных, новостных или технических текстов, художественные произведения характеризуются высокой степенью стилистической и семантической сложности. Нелинейность повествования, образность, метафоричность и стилистические приёмы делают задачу написания краткого содержания особенно трудоёмкой. Ограниченное контекстное окно современных моделей дополнительно осложняет работу с длинными произведениями.

Теоретически автоматическое реферирование может выполняться двумя основными способами: извлекающим (выбор ключевых фрагментов текста) и абстрактивным (генерация нового текста на основе содержания оригинала). Для художественной литературы более уместен второй подход, поскольку он позволяет передать смысл и стиль произведения, не нарушая его целостности.

% ================= НАБОР ДАННЫХ =================
\section*{Набор данных}
На момент начала исследования не существовало открытых и репрезентативных корпусов, предназначенных специально для задачи реферирования художественных текстов на русском языке. Был сформирован собственный набор данных, включающий:
\begin{itemize}
  \item более 600 пользовательских пересказов с ресурса «Народный Брифли»;
  \item исходные произведения из электронной библиотеки LibRuSec (публичное достояние или тексты с разрешения правообладателей);
\end{itemize}
Тексты аннотаций проходили автоматическую очистку от HTML‑тегов, комментариев и служебных пометок с помощью LLM~Meta--Llama~3--70B--Instruct. Затем производился поиск по датасету LibRuSec и собиралась коллекция, состоящая из пар "текст книги - аннотация".

% ================= ОЦЕНИВАНИЕ =================
\section*{Оценивание методов}
На момент начала исследования в русскоязычном сегменте отсутствовали открытые и репрезентативные наборы данных, специально предназначенные для задачи автоматического реферирования художественных текстов.
В отличие от аналогичных англоязычных проектов, которые уже включали крупные корпуса литературных произведений с высококачественными аннотациями, для русскоязычных текстов подобные ресурсы не были представлены. 
Это существенно затрудняло объективное тестирование и сравнение эффективности методов автоматического реферирования в условиях художественных произведений на русском языке.

С целью проведения экспериментов и оценки различных подходов к генерации аннотаций был создан собственный корпус, состоящий из художественных текстов и соответствующих кратких пересказов. 
В качестве источника для аннотаций был выбран ресурс «Народный Брифли»~\cite{Briefly} — платформа, где пользователи публикуют краткие пересказы литературных произведений. 
Несмотря на вариативность качества и стиля пользовательских аннотаций и наличие нерелевантной информации, такой как учебные вопросы или редакторские замечания, после тщательной предварительной обработки удалось получить достаточно надёжный и чистый набор данных.

Художественные тексты были отобраны из электронной библиотеки LibRuSec — одного из крупнейших русскоязычных ресурсов художественной литературы, содержащего свыше 400 тысяч текстов. 
Отбор произведений осуществлялся на основании наличия аннотаций на выбранном ресурсе. Каждый текст проходил автоматическую предварительную обработку: удалялась метаинформация (например, заголовки, описания глав и технические вставки), 
после чего текст форматировался в единый стандартизированный вид, подходящий для дальнейшего использования в моделях.
Важно отметить, что при создании корпуса использовались только тексты, находящиеся в общественном достоянии или распространяемые свободно с разрешения правообладателей, что обеспечивает соблюдение требований авторского права.

Для иллюстрации структуры сформированного корпуса ниже представлен пример пары «фрагмент художественного текста — соответствующая аннотация».
% ================= ПРИМЕНЕНИЕ МЕТОДОВ =================
\section*{Применение методов}
Экспериментальная часть включает проверку базовых стратегий (иерархическая, итеративная, «чертёжная») и двух усовершенствований.

\subsection*{Влияние предварительной очистки}
Очистка исходного текста от технических артефактов положительно сказалась на ROUGE--L и BERTScore для моделей Meta\-Llama~3\-70B и Ru\-adapt\-Qwen2.5\-32B\-Pro\-Beta, повысив также полноту покрытия ключевых вопросов.

\subsection*{Сравнение базовых методов}
Иерархический и итеративный подходы продемонстрировали сопоставимое качество (ROUGE--L\,$\approx$\,0.48, BERTScore\,$\approx$\,0.70), заметно превосходя псевдо‑генерацию без доступа к исходному тексту.

\subsection*{Иерархический метод с фильтрацией узлов}
Добавление фильтра по семантической близости (порог~0.85) позволило убрать избыточные фрагменты и ускорить генерацию на длинных текстах в среднем на~15\,\%.

\subsection*{«Чертёжный» метод с кластеризацией вопросов}
Кластеризация эмбеддингов вопросов алгоритмом \textit{k}-means и последующее сэмплирование 10--30 вопросов из каждого кластера уменьшили число обращений к модели и ускорили работу метода в~1.22 раза без потери качества.

\section*{Оценивание методов}

В таблице \ref{tab:results_models} приведены сравнительные результаты работы описанных выше методов генерации кратких пересказов художественных текстов.
Для каждой из исллдованных моделей измерялись метрики качеств  и время выполнеия в зависимости от метода: базовые чертежный и иерархический методы, а также их усовершенствованные версии \- 
чертежный метод с кластеризацией вопросов и иерархический метод с фильтрацией узлов.

\begin{table}[ht]
\centering
\small                       % уменьшили шрифт (можно \footnotesize)
\setlength{\tabcolsep}{4pt}  % сузили горизонтальные отступы между колонками
%\renewcommand{\arraystretch}{0.9} % можно ещё уплотнить строки

\caption{Результаты по методам и моделям}
\label{tab:results_models}

\begin{tabular}{llcccc}      % 2 текстовых столбца + 4 числовых
\toprule
Модель & Метрики & Чертежный & \shortstack{Чертежный\\с кластеризацией} & Иерархический & \shortstack{Иерархический\\с фильтрацией} \\
\midrule
\multirow{2}{*}{RuadaptQwen2.5-7B-Lite-Beta}
 & bertscore & 58.7 ± 3.8 & 57.7 ± 3.7 & 59.6 ± 3.5 & 59.3 ± 3.4 \\
 & rouge-l & 14.1 ± 4.8 & 12.2 ± 4.5 & 14.4 ± 4.3 & 13.8 ± 4.3 \\
 & time & 159.13 & 77.30 & 106.18 & 79.58 \\
\midrule
\multirow{2}{*}{RuadaptQwen3-32B-Instruct-v2}
& bertscore & 56.8 ± 5.8 & 53.7 ± 5.2 & 55.6 ± 3.4 & 55.7 ± 3.6 \\
& rouge-l   & 10.4 ± 4.5 & 7.6 ± 3.6 & 10.2 ± 2.9 & 9.9 ± 2.5 \\
& time & 201.88 & 140.41 & 182.41 & 147.32 \\
\midrule
\multirow{2}{*}{yagpt5lite}
 & bertscore & 61.0 ± 3.6 & 61.4 ± 3.2 & 62.3 ± 3.2 & 62.1 ± 3.3 \\
 & rouge-l & 15.8 ± 5.2 & 14.0 ± 4.4 & 16.7 ± 5.0 & 16.5 ± 4.7 \\
 & time & 98.45 & 27.06 & 24.97 & 24.34 \\
\midrule
\multirow{2}{*}{Qwen3-235B-A22B}
 & bertscore & 61.6 ± 3.3 & 59.3 ± 3.4 & 61.2 ± 3.0 & 60.9 ± 2.7 \\
 & rouge-l & 15.8 ± 4.5 & 12.2 ± 3.6 & 14.9 ± 4.0 & 14.8 ± 3.7 \\
 & time & 200.30 & 149.11 & 103.49 & 83.06 \\
\midrule
\multirow{2}{*}{DeepSeek V3}
 & bertscore & 58.0 ± 4.0 & 58.4 ± 3.6 & 60.0 ± 3.1 & 60.0 ± 2.9 \\
 & rouge-l & 12.6 ± 4.6 & 11.2 ± 3.9 & 13.7 ± 3.9 & 13.5 ± 3.7 \\
 & time & 315.67 & 132.60 & 196.77 & 147.21 \\
\midrule
\multirow{2}{*}{tpro}
 & bertscore & 59.0 ± 4.9 & 58.2 ± 3.7 & 59.4 ± 3.0 & 59.5 ± 3.3 \\
 & rouge-l & 14.7 ± 4.9 & 11.8 ± 3.9 & 13.8 ± 3.1 & 13.5 ± 3.0 \\
 & time & 259.35 & 161.33 & 276.45 & 230.21 \\ 
\bottomrule
\end{tabular}
\end{table}

Как видно из таблицы \ref{tab:results_models}, модифицированные варианты методов действительно существенно ускоряют обработку: среднее время генерации сокращается в 1.5-3 раза в зависимости от модели (например, DeepSeek V3: 315.67 $\implies$ 132.60)
Однако выигрыш по скорости сопровождается умеренным снижением качества: падение BERTScore и ROUGE-L чаще всего укладывается в 1-2 пункта и находится в пределах стандартных отклонений.
% -------------------------------------------------------



%-------РЕФЕРЕНСЫ---------
\begin{thebibliography}{99}
\bibitem{Briefly}
\textit{Народный Брифли.}  
Электронная библиотека кратких пересказов литературных произведений.  
\url{https://wiki.briefly.ru/} (дата обращения: 16.07.2025).

\end{thebibliography}

\renewcommand\refname{References}


\begin{thebibliography}{99}
\bibitem{Briefly_e}
\textit{Народный Брифли.}  
Электронная библиотека кратких пересказов литературных произведений.  
\url{https://wiki.briefly.ru/} (дата обращения: 16.07.2025).

\end{thebibliography}

\end{document}